% -*-coding: utf-8 -*-

\zp{Překvápko}{Hop Trop}

\zs
\Ch{C}{Zaslech'} \Ch{Ami}{jsem, že} \Ch{Dmi}{tuto}\Ch{G7}{vě v} \Ch{C}{Praze} \Ch{Ami}{pět na} \Ch{Dmi}{Smícho}\Ch{G7}{vě}

\Ch{C}{někdo} \Ch{Ami}{střelí} \Ch{Dmi}{indi}\Ch{G7}{ánskou} \Ch{C}{loď,} \Ch{C7}{}

hned \Ch{F}{napadlo} mě \Ch{C7}{bleskově,} že \Ch{F}{po důkladný} \Ch{C7}{opravě}

s ní \Ch{F}{překvapím} svou \Ch{C7}{milovanou} \Ch{F}{choť.} \Ch{G}{}
\ks

\zs
Já nezaváhal chviličku, co chtěl by za tu věcičku,

hned s majitelem vyjednávám sumu.

Říkal, že díky požáru má jen tu loď a kytaru

a k ní mu chybí aspoň doušek rumu.
\ks

\zs
Tak do nejbližší Jednoty já začal nosit hodnoty,

byly toho dvě narvaný tašky,

už mám tě, moje kocábko, pro moji drahou překvápko

mě stálo tenkrát čtyřicet dvě flašky.
\ks

\zr
\Ch{G}{Poplujem} \Ch{D7}{spolu} tam dolů tou peřejí,

přestože \Ch{G}{vodákům} v ČSD nepřejí,

řeka nám \Ch{C}{píseň} \Ch{Cmi}{bude} \Ch{G}{hrát,}

že lodě \Ch{D7}{nechtěj'} nikde \Ch{G}{brát.}
\kr

\zs
Den nato jsem pak z moskviče, co půjčujou nám rodiče,

vyndal dva tři nepotřebný díly,

co neudělat pro lásku, já pro tebe, můj vobrázku,

se nerozmejšlím nikdy ani chvíli.
\ks

\zs
Šoupnul jsem do svý aktovky dvě fungl nový mlhovky,

pár součástek a taky litřík Arvy,

vše vyměnil za lepidlo, pět latí, šrouby, tužidlo

a hlavně pikslu bleděmodrý barvy.
\ks

\zs
Už starodávná tradice říká, že právě Lužnice

vodáků je každoročně Mekkou,

v den D u mostu v Suchdole: \uv{No to je voda, tý vole!,}

my zírali nad rozvodněnou řekou.
\ks

\zs
Aby ti, co koukaj' okolo, nám neříkali \uv{prďolo,}

my do svý lodi flegmaticky vlezli,

že přišlo velký koupání a vo šutráky drncání,

to nevadí, no aspoň jsme se svezli!
\ks

\zr
Koukáme spolu tam dolů tou peřejí na trosku lodě, co volně proud nese ji,

řeka nám píseň bude hrát, zbyla nám žížeň akorát.
\kr

\kp
