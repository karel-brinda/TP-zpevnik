% -*-coding: utf-8 -*-

\zp{Hotel Hillary}{Poutníci}

\zs
Tvař se \Ch{Ami}{trochu} nostalgicky, už tě nikdy nepotkám, \Ch{G}{}

\Ch{F}{máš} to jistý \Ch{G}{provždycky}, nastav \Ch{Ami}{uši vzpomínkám,}

jak tě znám, i v tuhle chvíli měl bys řeči peprný,  \Ch{G}{}

jak \Ch{F}{tenkrát}, když nám \Ch{G}{tvrdili}, že je \Ch{Ami}{vítr} stříbrný.  \Ch{G}{}
\ks

\zr
A \Ch{F}{tváře} měli kožený, my jim zdrhli z průvodu,

zaho\Ch{Dmi}{dili lampióny} a \Ch{D}{našli} hospodu,

ale \Ch{F}{taky} Jacquese Brela a s ním smutek z cizích vin

a \Ch{Dmi}{žádostivost} těla a pak \Ch{D}{radost} z volovin,

a ta nám \Ch{Ami}{zbejvá}.
\kr

\zs
Po večerech pro diváky dělali jsme kašpary,

pak na zemi dva spacáky -- náš Hotel Hillary,

slavný sliby jsme už znali, i to, jak se neplní,

a cenzoři nám kázali o správným umění.
\ks

\zr  \kr

\zs
A tak válčím s nostalgií, bují ve mně jako mech

a pořád všechno slibují starý hesla na domech,

ty jsi splatil všechny dluhy, i za Hotel Hillary,

já vyhážu ty černý stuhy funebrákům navzdory.
\ks

\zr
Vždyť mají tváře kožený, my jim zdrhnem z průvodu,

zahodíme lampióny a najdem hospodu,

a tam svýho Jacquese Brela a s ním smutek z cizích vin

a žádostivost těla a pak radost z volovin,

/: a ta nám zbejvá. :/
\kr

\kp





