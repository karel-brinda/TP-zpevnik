% -*-coding: utf-8 -*-

\zp{Básnířka}{Jarek Nohavica}
\zs
\Ch{G}{Mladičká} \Ch{D}{bás}nířka s \Ch{C}{korálky} \Ch{D}{nad} kotní\Ch{G}{ky} \Ch{D}{} \Ch{C}{} \Ch{D}{}

\Ch{G}{bouchala} \Ch{D}{na} dvířka \Ch{C}{paláce} \Ch{D}{poe}ti\Ch{Emi}{ky.}
\ifdefined\TPBAND
	(\Ch{G}{ba}\Ch{B}{ss}\Ch{C}{a)}
\fi

\Ch{G}{S někým} se \Ch{C}{vyspa}\Ch{G}{la}, někomu \Ch{C}{neda}\Ch{G}{la},
láska jako \Ch{D}{hobby,}
\ifdefined\TPBAND
	(\Ch{E}{ba}\Ch{F#}{ssa)}
\fi

\Ch{G}{pak} o tom \Ch{D}{nap}sala \Ch{C}{sonetu} na \Ch{D}{čtyř}i 
\Ch{G}{doby.}  \Ch{D}{} \Ch{C}{} \Ch{D}{}
\ks
\zs
Svoje srdce skloňovala podle vzoru Ferlinghetti,

ve vzduchu nechávala viset vždy jen půlku věty.

Plná tragiky, plná mystiky,
plná splínu,

tak jí to otiskli v jednom magazínu.
\ks
\zs
Bývala viděna v malém baru u Rozhlasu,

od sebe kolena a cizí ruka kolem pasu.

Trochu se napila, trochu se opila
na účet redaktora,

za týden na to byla hvězdou mikrofóra.
\ks
\zs
Pod paží nosila rozepsané rukopisy,

ráno se budila vedle záchodové mísy.

Múzou políbená, životem potřísněná,
plná zázraků

a pak ji vyhodili z gymplu i z baráku.
\ks
\zs
Šly řeči okolím, že měla něco se esenbáky,

ať bylo cokoliv, přestala věřit na zázraky.

Cítila u srdce, jak po ní přešla
železná bota,

tak o tom napsala sonet ze života.
\ks
\zs
Pak jednou v pondělí, přišla na koncert na koleje

a hlasem pokorným prosila o text Darmoděje.

Pero si vzala a pak se dala,
tichounce do pláče

/: a její slzy kapaly na její mrkváče. :/
\ks
\kp





