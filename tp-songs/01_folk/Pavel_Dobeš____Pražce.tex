% -*-coding: utf-8 -*-

\zp{Pražce}{Pavel Dobeš}

\zs
\Ch{C}{Házím} tornu na svý záda, feldflašku a \Ch{G7}{sumky,}

navštívím dnes kamaráda z železniční průmky,

vždyť je \Ch{C}{jaro,} zapni si kšandy,
pozdravuj \Ch{G7}{vlaštovky} a muziko, ty \Ch{C}{hraj.}
\ks

\zs
Vystupuji z vlaku, který mizí v dálce,

stojím v České Třebové a všude kolem pražce,

vždyť je jaro, zapni si kšandy,
pozdravuj vlaštovky a muziko, ty hraj.
\ks

\zs
Pohostil mě slivovicí, představil mě Mařce,

posadil mě na lavici z dubového pražce,

vždyť je jaro, zapni si kšandy,
pozdravuj vlaštovky a muziko, ty hraj.
\ks

\zs
Provedl mě domem -- nikde kousek zdiva,

všude samej pražec, jen Máňa byla živá.

To je to jaro, zapni si kšandy,
pozdravuj vlaštovky a muziko, ty hraj.
\ks

\zs
Plakáty nás informují: \uv{Přijď pracovat k dráze,

pakliže ti vyhovují rychlost, šmír a saze,}

vždyť je jaro, zapni si kšandy,
pozdravuj vlaštovky a muziko, ty hraj.
\ks

\zs
A jestliže jsi labužník a přes kapsu se praštíš,

upečeš i krávu na železničních pražcích,

vždyť je jaro, zapni si kšandy,
pozdravuj vlaštovky a muziko, ty hraj.
\ks

\zs
A naučíš se skákat tak, jak to umí vrabec,

když na nohu si pustíš železniční pražec,

vždyť je jaro, zapni si kšandy,
pozdravuj vlaštovky a muziko, ty hraj.
\ks

\zs
Když má děvče z Třebové rádo svého chlapce,

posílá mu na vojnu železniční pražce,

vždyť je jaro, zapni si kšandy,
pozdravuj vlaštovky a muziko, ty hraj.
\ks

\zs
A když děti zlobí, tak hned je doma mazec,

Děda Mráz jim nepřinese ani jeden pražec,

vždyť je jaro, zapni si kšandy,
pozdravuj vlaštovky a muziko, ty hraj.
\ks

\zs
Před děvčaty z Třebové chlubil jsem se silou,

pozvedl jsem pražec, načež odvezli mě s kýlou,

vždyť je jaro, zapni si kšandy,
pozdravuj vlaštovky a muziko, ty hraj.
\ks

\zs
Pamatuji pouze ještě operační sál,

pak praštili mě pražcem a já jsem tvrdě spal

a bylo jaro, zapni si kšandy,
lítaly vlaštovky a zelenal se háj.
\ks

\kp


















