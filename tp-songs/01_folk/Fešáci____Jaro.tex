% -*-coding: utf-8 -*-

\zp{Jaro}{Fešáci}
\zs
\Ch{Ami}{My} čekali \Ch{C}{jaro} a \Ch{G}{zatím} přišel \Ch{Ami}{mráz,}
tak strašlivou \Ch{C}{zimu} ne\Ch{G}{zažil} nikdo z \Ch{Ami}{nás,}

z těžkých černých \Ch{C}{mraků} se \Ch{G}{stále} sypal \Ch{Ami}{sníh}
a vánice \Ch{C}{sílí} v po\Ch{G}{ryvech} ledo\Ch{Ami}{vých.}

Z \Ch{C}{chýší} dřevo mizí a \Ch{G}{mouky} ubývá,
\Ch{Dmi}{do sýpek} se raději už \Ch{G}{nikdo} nedívá,

\Ch{C}{zvěř} z okolních lesů nám \Ch{G}{stála} u dveří
\Ch{Dmi}{a hladoví} ptáci při\Ch{G}{létli} za zvěří a stále \Ch{Ami}{blíž.}
\ks
\zs
Jednoho dne večer, to už jsem skoro spal,

když vystrašený soused na okno zaklepal:

\uv{Můj synek doma leží, v horečkách vyvádí,

já do města bych zašel, doktor snad poradí.}

Půjčil jsem mu koně a když sedlo zapínal,

dříve, než se rozjel, jsem ho ještě varoval:

\uv{Nejezdi naší zkratkou, je tam příkrej sráz

a v týhletý bouři tam snadno zlámeš vaz, tak neriskuj!}
\ks
\zs
Na to strašné ráno dnes nerad vzpomínám,

na tu hroznou chvíli, když kůň se vrátil sám,

trvalo to dlouho, než se vítr utišil,

na sněhové pláně si každý pospíšil.

Jeli jsme tou zkratkou až k místu, které znám,

kterým bych v té bouři nejel ani sám,

a pak ho někdo spatřil, jak tam leží pod srázem,

krev nám tuhla v žilách nad tím obrazem, já klobouk sňal.
\ks

\Ch{Ami}{Někdy} ten, kdo \Ch{C}{spěchá}, se \Ch{G}{domů} nevra\Ch{Ami}{cí...}
\kp





