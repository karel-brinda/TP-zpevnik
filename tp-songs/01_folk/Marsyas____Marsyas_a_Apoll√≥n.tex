% -*-coding: utf-8 -*-

\zp{Marsyas a Apollón}{Marsyas}

\zs
<D>Ta krásná dívka, <Emi>co se bojí o <A>svoji <Asus4>krásu,

<D>Athéna <Emi7>jméno má <G>za starých <D>dávných časů,

<D>odhodí flétnu, <Emi>hrát nejde s <A>nehybnou <Asus4>tváří,

<D>ten, kdo ji <Emi7>najde dřív, tomu se <D>přání zmaří.
\ks

\zr
<Hmi>Tak i Marsyas <A>zmámen flétnou <Hmi>věří, že <G>musí přetnout

/: <D>jedno pravidlo, <Emi>sázku a <G>hrát <D>líp <A>než <D>bůh. :/
\kr

\zs
Bláznivý nápad, snad nejvýš Marsyas míří,

Apollón souhlasí, oba se s trestem smíří,

král Midas má říct, kdo je lepší, Apollón zpívá,

o život soupeří, jen jeden vítěz bývá.
\ks

\zr
Tak si Marsyas mámen flétnou věří a musí přetnout

/: jedno pravidlo, sázku a hrát líp než bůh. :/
\kr

\zs
Obrátí nástroj, už ví, že nebude chválen,

prohrál a zápolí podveden vůlí krále,

sám v tichém hloučku, sám na strom připraví ráhno,

satyra k hrůze všech zaživa z kůže stáhnou.
\ks

\zr
Tak si Apollón změřil síly, každý se musel mýlit,

/: nikdo nemůže kouzlit a hrát líp než bůh. :/
\kr

\zs
Ta krásná dívka, co se bála o svoji krásu,

dárkyně moudrosti za starých dávných časů,

teď v tichém hloučku, v jejích rukou úroda, spása,

Athéna jméno má, chybí jí tvář a krása.
\ks

\zr
Dy dy dy...
\kr

\kp
