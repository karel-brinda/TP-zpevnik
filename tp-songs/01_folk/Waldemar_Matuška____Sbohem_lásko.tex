% -*-coding: utf-8 -*-

\zp{Sbohem, lásko}{Waldemar Matuška}

\zs
Ať bylo \Ch{C}{mně} i \Ch{F}{jí tak} \Ch{G}{šestnáct} \Ch{C}{let,} \Ch{F}{} \Ch{G}{}

zeleným \Ch{C}{údolím} \Ch{Ami}{jsem si ji} \Ch{Dmi}{ved',} \Ch{G7}{}

\Ch{C}{byla krásná,} to \Ch{C7}{vím,} a já měl \Ch{F}{strach,} jak \Ch{Fmi}{říct,}

když na řa\Ch{C}{sách} slzu \Ch{G7}{má velkou} jako \Ch{C}{hrách:} \Ch{F}{} \Ch{C}{}
\ks

\zr
\uv{\Ch{C7}{Sbohem,} \Ch{F}{lásko,} nech mě \Ch{Fmi}{jít, nech mě} \Ch{Emi}{jít, bude} \Ch{Ami}{klid,}

žádnej \Ch{Dmi}{pláč už nespra}\Ch{G7}{ví ty mý} \Ch{C}{nohy} toula\Ch{C7}{vý,}

já tě \Ch{F}{vážně} měl moc \Ch{Fmi}{rád, co ti} \Ch{Emi}{víc můžu} \Ch{Ami}{dát?}

Nejsem \Ch{Dmi}{žádnej ide}\Ch{G7}{ál, tak nech mě} \Ch{C}{jít zas} \Ch{F}{o dům} \Ch{C}{dál.}}
\kr

\zs
A tak šel čas, a já se toulám dál,

v kolika údolích jsem takhle stál,

hledal slůvka, co jsou jak hojivej fáč, bůhví,

co jsem to zač, že přináším všem jenom pláč.
\ks

\zr\kr

Rec: Já nevím, kde se to v člověku bere -- ten neklid, co ho tahá z místa na 
místo, co ho nenechá, aby byl sám se sebou spokojený jako většina ostatních, 
aby se usadil, aby dělal jenom to, co se má, a říkal jenom to, co se od něj 
čeká, já prostě nemůžu zůstat na jednom místě, nemůžu, opravdu, fakt.

\zr\kr

\kp
