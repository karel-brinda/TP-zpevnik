% -*-coding: utf-8 -*-

\zp{Dvě spálený srdce (Nagasaki, Hirošima)}{Karel Plíhal}

\zs
\Ch{C}{Tramvají} \Ch{G}{dvojkou} \Ch{F}{jezdíval} jsem \Ch{G}{do Žide}\Ch{C}{nic,} \Ch{G}{} \Ch{F}{} \Ch{G}{}

z \Ch{C}{tak veliký} \Ch{G}{lásky} \Ch{F}{většinou} \Ch{G}{nezbyde} \Ch{Ami}{nic,}

z \Ch{F}{takový} \Ch{C}{lásky} \Ch{F}{jsou kruhy} \Ch{C}{pod oči}\Ch{G}{ma}

a \Ch{C}{dvě spálený} \Ch{G}{srdce} -- \Ch{F}{Nagasaki,} \Ch{G}{Hirošima.} \Ch{C}{} \Ch{G}{} \Ch{F}{} \Ch{G}{}
\ks

\zs
Jsou jistý věci, co bych tesal do kamene,

tam, kde je láska, tam je všechno dovolené,

a tam, kde není, tam mě to nezajímá,

jó, dvě spálený srdce -- Nagasaki, Hirošima.
\ks

\zs
Já nejsem svatej, ani ty nejsi svatá,

ale jablka z ráje bejvala jedovatá,

jenže hezky jsi hřála, když mi někdy bylo zima,

jó, dvě spálený srdce -- Nagasaki, Hirošima.
\ks

\zs
= 1. + /: a dvě spálený srdce -- Nagasaki, Hirošima. :/
\ks

\kp
