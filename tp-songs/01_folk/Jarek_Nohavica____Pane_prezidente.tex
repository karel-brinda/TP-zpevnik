% -*-coding: utf-8 -*-

\zp{Pane prezidente}{Jarek Nohavica}

\zs
Pane <D>prezidente, ústavní <Emi>činiteli,

píšu <A7>vám dopis, že moje děti na mě <D>zapomněly.

Jde o syna Karla a mladší <Emi>dceru Evu,

rok už <A7>nenapsali, rok už nepřijeli <D>na návštěvu.
\ks

\zr
Vy to <G>pochopíte, vy přece <Ami>všechno víte,

vy se <D7>poradíte, vy to vyřešíte, vy mě <G>zachráníte.

Pane prezidente, já chci jen <Ami>kousek štěstí,

pro co <D7>jiného jsme přeci zvonili klíčema <G>na náměstí. <A7>
\kr

\zs
Pane prezidente, a ještě stěžuju si,

že mi podražili pivo, jogurty, párky i trolejbusy

i poštovní známky i bločky na poznámky

i telecí plecko, Řecko i Německo -- no prostě všecko.
\ks

\zr\kr

\zs
Pane prezidente mé České republiky,

oni mě propustili na hodinu z mý fabriky,

celých třicet roků všecko bylo v cajku,

teď přišli noví mladí a ti tu řádí jak na Klondiku.
\ks

\zr
Vy to pochopíte, vy se mnou soucítíte...
\kr

\zs
Pane prezidente, moje Anežka kdyby žila,

ta by mě za ten dopis, co tu teď píšu, patrně přizabila,

řekla by: \uv{Jaromíre, chováš se jak malé děcko,

víš co on má starostí s celú tu Evropu, s vesmírem

a vůbec všecko!}
\ks

\zr
Ale vy mě pochopíte...
\kr

Pane <D>prezidente!

\kp
