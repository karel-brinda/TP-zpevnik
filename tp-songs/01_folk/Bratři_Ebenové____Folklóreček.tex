% -*-coding: utf-8 -*-

\zp{Folklóreček}{Bratři Ebenové}

Zahrajte mi muzikanti zvesela, zvesela!

\zs
\Ch{D}{Slunečko} vychází nad vysokú \Ch{C}{horú},

\Ch{G}{počítám,} že \Ch{Emi}{večer} \Ch{G}{pude} zase \Ch{A}{do}-\Ch{D}{lu.}

Večer pude dolu, zitra pujde znovu,

takhle tu pisničku máme hned hotovú.
\ks

\zr
\Ch{A}{Nejvíc} mě ten folkloreček {do}{jí}\Ch{C}{má,}

\Ch{G}{když} se hraje \Ch{Emi}{elekt}ricky, \Ch{G}{elektricky} s \Ch{Emi}{bi-}\Ch{A}{cí}-\Ch{D}{ma}.
\kr

\zs
Šavlička se blýská na severní stranu,

košulenku Jano ma vypasovanú.

Jede na koničku, jede do vesničky,

za klobúčkem perko a v ruce husličky.
\ks

\zr  \kr

\zs
Zahraj mi Janičku, zahraj na húsličky,

potěš ty tej mojej přesmutnej dušičky.

Zahrál mi Janiček, zahrál na húsličky,

hned z toho natočil dvě pěkné destičky.

\ks

\zr  \kr

\zs
Kdo si to cedečko doma vypaluje,

ten moje srdečko tuze zarmucuje.

Komu tu cedečku najdu na pisičku,

tomu svu šavličku useknu ručičku.
\ks

\zr  \kr

\kp






