% -*-coding: utf-8 -*-

\zp{Už koníček pádí}{Waldemar Matuška}

\zs
Znám \Ch{G7}{zem plnou} \Ch{C}{mlíka} a buclatejch \Ch{G}{krav,}

kde proud řeky \Ch{D7}{stříká} na dřevěnej \Ch{G}{splav.}

Mám \Ch{G7}{jediný} \Ch{C}{přání,} snům ostruhy \Ch{G}{dát}

a pod známou \Ch{D7}{strání} zas kuličky \Ch{G}{hrát.}
\ks

\zr
Už koníček pádí a zůstane stát až v tý zemi mládí, kde já žiju rád.

A slunce tam pálí a pořád je máj a ceny jsou stálý a lidi se maj'.
\kr

\zs
Kde všechno je známý, zvuk tátovejch bot a buchty mý mámy a natřenej plot.

Z něj barva už prejská, ale mně je to fuk, já, když se mi stejská, jsem jak malej kluk.
\ks

\zr\kr

\zs
V tý zemi jsou lípy a ve květech med a u sudů pípy a u piva led.

A okurky v láku a cestovní ruch a hospod jak máku a holek jak much.
\ks

\zr
A slunce tam pálí a pořád je máj a ceny jsou stálý a lidi se maj'.

Už koníček pádí a zůstane stát, až v tý zemi mládí, kde žiju tak rád.
\kr

\kp
