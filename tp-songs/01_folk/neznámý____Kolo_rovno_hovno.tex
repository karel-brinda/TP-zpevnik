% -*-coding: utf-8 -*-

\zp{Kolo, rovno, hovno}{(neznámý)}

\zs
\Ch{D}{Kolo}, rovno, hovno, jak \Ch{A7}{se to} rýmu\Ch{D}{je,}
kolo, rovno, hovno, jak \Ch{G}{se} to rýmu\Ch{D}{je!}

kolo jede \Ch{A7}{rovno,} rovno přes to \Ch{D}{hovno,}
rovno přes to hovno, tak \Ch{A7}{se to} rýmu\Ch{D}{je.}
\ks
\zs
/: Kostel, prdel, papír, jak se to rýmuje? :/

/: Kostel, to je víra, prdel, to je díra, papírem se utírá,

tak se to rýmuje. :/
\ks
\zs
/: Šnytlík, pytlík, rychlík, jak se to rýmuje? :/

/: Když tě svrbí pytlík, dej si na něj šnytlík a pojedeš jak rychlík,

tak se to rýmuje. :/
\ks
\zs
/: Puška, voda, blbost, jak se to rýmuje? :/

/: Puškou, tou se míří, voda, ta se víří a blbost, ta se šíří,

tak se to rýmuje. :/
\ks
\zs
/: Tygři, hasiči, sulc, jak se to rýmuje? :/

/: Tygři, to jsou šelmy, hasiči maj helmy a sulc se třese velmi,

tak se to rýmuje. :/
\ks
\zs
/: Hospoda, náves, voda, jak se to rýmuje? :/

/: V hospodě se pije, na návsi se blije a voda všechno smyje,

tak se to rýmuje. :/
\ks
\zs
/: Prádlo, peří, škola, jak se to rýmuje? :/

/: Prádlo, to se pere, peří, to se dere a na školu se vzpomíná,

tak se to rýmuje. :/
\ks
\zs
/: Hraběnka, Monča, postel, jak se to rýmuje? :/

/: Hraběnka je hrdá, Monča zase Tvrdá a v posteli se narodil spisovatel Drda. :/
\ks


\kp
