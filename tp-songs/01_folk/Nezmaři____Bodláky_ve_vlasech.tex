% -*-coding: utf-8 -*-

\zp{Bodláky ve vlasech}{Nezmaři}

\zs
Do vla<G>sů bláznivej <Emi>kluk mi <Ami>bodláky <D>dával, 

<Emi>za tuhle <C>kytku pak <F>všechno chtěl <D>mít,

svateb<G>ní menu<D>et mi <Ami>na stýblo <H7>hrával, 

<C>že prej se <D>musíme <G>vzít. <D7>
\ks

\zs
Zelený, voňavý dva prstýnky z trávy,

copak si holka víc může tak přát?

Doznívá menuet, čím dál míň mě baví

na tichou poštu si hrát.
\ks

\zr
<Emi>Bez bolesti divný <D>trápení, <Emi>suchej pramen těžko <D>pít,

<G>zbytečně <C>slova do <Cdim>kamení <H7>sít.

<G7>Na košili našich <C>zvyků, <F>vlajou nitě od 
knof<B>líků,

<Eb>jeden je \uv{<C>muset} a <Eb>druhej je \uv{<G>chtít.} <G7>
\kr

\zs
Do vlasů bláznivej kluk ti bodláky dával,

za tuhle kytku pak všechno chtěl mít,

svatební menuet ti na stýblo hrával,

my dva se musíme vzít.
\ks

\zs
Zelený, voňavý dva prstýnky z trávy,

nejsem si jistej, že víc umím dát,

vracím se zkroušenej, ale dobrý mám zprávy

o tom, že dál tě mám rád.
\ks

\zs
Zelený, voňavý dva prstýnky z trávy,

copak si my dva víc můžeme přát?

Dál nám zní menuet, tím míň nás baví

na tichou poštu si hrát.



Zelený voňavý .....

Prstýnky voňavý .....

Z trávy zelený ..... 
\ks
\kp

