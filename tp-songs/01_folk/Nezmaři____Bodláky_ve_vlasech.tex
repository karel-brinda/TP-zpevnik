% -*-coding: utf-8 -*-

\zp{Bodláky ve vlasech}{Nezmaři}

\zs
\Ch{G}{Do} vlasů bláznivej kluk \Ch{Emi}{mi} bodláky \Ch{Ami}{dával}\Ch{D}{} 

\Ch{Emi}{za} tuhle \Ch{C}{kytku} pak \Ch{F}{všechno} chtěl mít\Ch{D}{} 

\Ch{G}{svatební} menuet \Ch{D}{mi} na \Ch{Ami}{stýblo} hrával\Ch{H7}{} 

\Ch{C}{že} prej se \Ch{D}{musíme} vzít. \Ch{G}{} \Ch{D7}{}
\ks

\zs
Zelený voňavý dva prstýnky z trávy

copak si holka víc může tak přát.

Doznívá menuet čím dál míň mě baví

na tichou poštu si hrát.
\ks

\zr
\Ch{Emi}{Bez} bolesti divný \Ch{D}{trápení}, \Ch{Emi}{suchej} pramen těžko \Ch{D}{pít},

\Ch{G}{zbytečně} \Ch{C}{slova} do kame\Ch{Cdim}{ní} \Ch{H7}{sít}.

\Ch{G7}{Na} košili našich \Ch{C}{zvyků}, \Ch{F}{vlajou} nitě od knof\Ch{B}{líků}.

\Ch{D}{jeden} \Ch{C}{je} Muset \Ch{D}{a} druhý \Ch{G}{Chtít}. \Ch{G7}{}
\kr

\zs
Do vlasů bláznivej kluk ti bodláky dával

za tuhle kytku pak všechno chtěl mít

svatební menuet ti na stýblo hrával

my dva se musíme vzít.
\ks

\zs
Zelený voňavý dva prstýnky z trávy

nejsem si jistej, že víc umím dát

Vracím se zkroušenej, ale dobrý mám zprávy

o tom že dál tě mám rád.
\ks

\zs
Zelený voňavý dva prstýnky z trávy

copak si my dva víc můžeme přát.

Dál nám zní menuet a tím míň nás baví

na tichou poštu si hrát.



Zelený voňavý .....

Prstýnky voňavý .....

Z trávy zelený ..... 
\ks
\kp

