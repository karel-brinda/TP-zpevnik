% -*-coding: utf-8 -*-

\zp{Bodláky ve vlasech}{Nezmaři}

\zs
Do vla\Ch{G}{sů} bláznivej \Ch{Emi}{kluk mi} \Ch{Ami}{bodláky} \Ch{D}{dával,} 

\Ch{Emi}{za tuhle} \Ch{C}{kytku} pak \Ch{F}{všechno} chtěl \Ch{D}{mít,}

svateb\Ch{G}{ní} menu\Ch{D}{et mi} \Ch{Ami}{na stýblo} \Ch{H7}{hrával,} 

\Ch{C}{že} prej se \Ch{D}{musíme} \Ch{G}{vzít.} \Ch{D7}{}
\ks

\zs
Zelený, voňavý dva prstýnky z trávy,

copak si holka víc může tak přát?

Doznívá menuet, čím dál míň mě baví

na tichou poštu si hrát.
\ks

\zr
\Ch{Emi}{Bez} bolesti divný \Ch{D}{trápení}, \Ch{Emi}{suchej} pramen těžko \Ch{D}{pít},

\Ch{G}{zbytečně} \Ch{C}{slova} do \Ch{Cdim}{kamení} \Ch{H7}{sít}.

\Ch{G7}{Na} košili našich \Ch{C}{zvyků}, \Ch{F}{vlajou} nitě od 
knof\Ch{B}{líků},

\Ch{Eb}{jeden} je \uv{\Ch{C}{muset}} a \Ch{Eb}{druhej} je \uv{\Ch{G}{chtít.}}  \Ch{G7}{}
\kr

\zs
Do vlasů bláznivej kluk ti bodláky dával,

za tuhle kytku pak všechno chtěl mít,

svatební menuet ti na stýblo hrával,

my dva se musíme vzít.
\ks

\zs
Zelený, voňavý dva prstýnky z trávy,

nejsem si jistej, že víc umím dát,

vracím se zkroušenej, ale dobrý mám zprávy

o tom, že dál tě mám rád.
\ks

\zs
Zelený, voňavý dva prstýnky z trávy,

copak si my dva víc můžeme přát?

Dál nám zní menuet, tím míň nás baví

na tichou poštu si hrát.



Zelený voňavý .....

Prstýnky voňavý .....

Z trávy zelený ..... 
\ks
\kp

