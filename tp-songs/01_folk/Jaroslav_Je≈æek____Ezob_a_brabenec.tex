% -*-coding: utf-8 -*-

\zp{Ezob a brabenec}{Jaroslav Ježek}

\zs
<C>Jednou z <Emi>lesa <Ami>domů se <Ami7>nesa, <F>mou<G7>drý <C>Ezop<G7>

<C>potkal <Emi>brabce, <Ami>který bra<Ami7>bence 
<F>má<G7>lem <A7>sezob.

<Dmi>Brabenec se <E>chechtá, <Dmi>Ezop se <D7>hned 
<G7>ptá,

<C>čemu <Emi>že se <Ami>na trávě v <Ami7>lese <F>prá<G>vě <Dmi>řeh<G7>tá.
\ks

\zs
\uv{<C>Já,} povídá <G7>brabenec, \uv{se taky <C>rád<B> hlasitě <A7>chechtám,

chech<D7>tám, když pupe<G#7>nec <G7>kyselinou <C>leptám. <D7> <Dmi7> <G7> 

<C>Vím,} totiž ten <G7>brabenec, \uv{mraveneč<C>ník <B>že 
se mě <A7>neptá,

<D7>neptá, pozře mě, ať se <G#7>chechtám -- 
<G7>nechech<C>tám.} <Dmi> <G7> <C> \ks

\zs
<G#>Kampak by to <Es7>došlo třeba s <G#>pouhou 
ponra<Es7>vou, 

<G>kdyby měla <D7>plakat, že je <Dmi>ptačí potra<G7>vou.

<C>Ty, ač nejsi <G7>brabenec, se taky <C>rád<B> hlasitě <A7>chechtej,

chech<D7>tej, a na svou <G#7>bídu si <G7>nezarep<C>tej.
\ks

\kp
