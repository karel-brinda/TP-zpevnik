% -*-coding: utf-8 -*-

\zp{Trubadúrská}{Karel Plíhal}

\zs
\Ch{Ami}{Od hradu} \Ch{Emi}{ke hradu} \Ch{Ami}{putujem,}
\Ch{C}{zpíváme} \Ch{G}{a} holky \Ch{E7}{muchlujem.}

/: \Ch{Ami}{Dřív} ji\Ch{Ami/G}{nam} \Ch{Fmaj7}{neje-}\Ch{Fmaj7/F#}{dem,}
\Ch{Ami}{dokud} tu \Ch{Emi}{poslední} \Ch{Ami}{nesvedem.} :/
\ks

\zs
Kytary nikdy nám neladí,
naše písně spíš kopnou než pohladí,

/: nakopnou zadnice
ctihodných měšťanů z radnice. :/
\ks

\zr
\Ch{G}{Hop} hej, je \Ch{C}{vese}\Ch{Emi7/H}{lo,
pan} \Ch{Fmaj7}{kníže} \Ch{D9/F#}{pozval} \Ch{G}{kejklíře,}

\Ch{G}{hop} hej, je \Ch{C}{vese}\Ch{Emi7/H}{lo,
dnes} \Ch{Fmaj7}{víta-}\Ch{Emi}{ní jsme} \Ch{Ami}{hosti.}

\Ch{G}{Hop} hej, je \Ch{C}{vese}\Ch{Emi7/H}{lo,
ač} \Ch{Fmaj7}{neda-}\Ch{D9/F#}{li nám} \Ch{G}{talíře,}

\Ch{G}{hop} hej, je \Ch{C}{vese}\Ch{Emi7/H}{lo,
pod} \Ch{Fmaj7}{stůl nám} \Ch{Emi}{hážou} \Ch{Ami}{kosti}.
\kr

\zs
Nemáme způsoby knížecí,
nikdy jsme nejedli telecí,

/: spáváme na seně,
proto vidíme život tak zkresleně. :/
\ks

\zs
A doufáme, že lidi pochopí,
že pletou si na sebe konopí,

/: že hnijou zaživa,
když brečí v hospodě u piva. :/
\ks

\zr  \kr

\zs
Ale jako bys lil vodu přes cedník,
je z tebe nakonec mučedník,

/: čekaj' tě ovace
a potom veřejná kremace. :/
\ks

\zs
Rozdělaj' pod náma ohýnky
a jsou z toho lidové dožínky.

Kdo to je tam u kůlu,
ale příliš si otvíral papulu.

Kdo to je tam u kůlu,
borec, za nás si otvíral papulu.
\ks

\zr  \kr

\zs
= 1.
\ks

\zs
To radši zaživa do hrobu,
než pověsit kytaru na skobu
a v hospodě znuděně čekat...
\ks

\kp




























































