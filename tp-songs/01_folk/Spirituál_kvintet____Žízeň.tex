% -*-coding: utf-8 -*-

\zp{Žízeň}{Spirituál kvintet}

\zs
Když ka\Ch{C}{pky} deště buší na \Ch{Fmaj7}{rozpále}\Ch{Ami}{nou} \Ch{G}{ze}\Ch{C}{m,}

já toužím celou duší dát \Ch{Fmaj7}{živou vodu} \Ch{C}{všem},

už v Knize knih je psáno: bez \Ch{Fmaj7}{vody} \Ch{Ami}{nel-}\Ch{G}{ze} \Ch{C}{žít},

však ne každému je dáno \Ch{Fmaj7}{z řeky} pravdy \Ch{C}{pít}.
\ks

\zr
Já mám ží\Ch{G}{zeň}, věčnou \Ch{C}{žízeň}, \Ch{C7}{}

stačí \Ch{F}{říct}, kde najdu vlá\Ch{C}{hu}

a zchladím \Ch{F}{žá}\Ch{C}{hu} páli\Ch{G}{vou},

ó, já mám žízeň, věčnou \Ch{C}{ží}\Ch{C7}{zeň,}

stačí \Ch{F}{říct}, kde najdu \Ch{C}{vláhu},

a zmizí \Ch{F}{ží}\Ch{C}{zeň}, \Ch{F}{ží}\Ch{C}{zeň}, \Ch{F}{ží}\Ch{C}{zeň}.
\kr

\zs
Stokrát víc než slova hladká jeden čin znamená,

však musíš zadní vrátka nechat zavřená,

mně čistá voda schází, mně chybí její třpyt,

vždyť z moře lží a frází se voda nedá pít.
\ks

\zr  \kr

\zs
Jak vytékají říčky zpod úbočí hor,

tak pod očními víčky já ukrývám svůj vzdor,

ten pramen vody živé má v sobě každý z nás

a vytryskne jak gejzír, až přijde jeho čas.
\ks

\zr  \kr  \zr  \kr

\kp
