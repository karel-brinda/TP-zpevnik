% -*-coding: utf-8 -*-

\zp{Okoř}{(neznámý)}

\zs
\Ch{D}{Na} Okoř je cesta jako žádná ze sta, \Ch{A7}{vroubená} je stroma\Ch{D}{ma.}

\Ch{D}{Když} du po ní v létě samoten ve světě, \Ch{A7}{sotva} pletu noha\Ch{D}{ma.}

\Ch{G}{Na} konci té cesty \Ch{D}{trnité} \Ch{E}{stojí} krčma jako \Ch{A7}{hrad,}

\Ch{D}{tam} zapadli trampi, (tam se trampi sešli,) hladoví a sešlí, \Ch{A7}{začli} sobě noto\Ch{D}{vat.}
\ks

\zr
\Ch{D}{Na} hradě Okoři \Ch{A7}{světla} už nehoří, \Ch{D}{Bílá} paní \Ch{A7}{šla} už dávno \Ch{D}{spát.}

\Ch{D}{Ona} měla ve zvyku \Ch{A7}{podle} svého budíku \Ch{D}{o půlnoci} \Ch{A7}{chodit} straší\Ch{D}{vat.}

\Ch{G}{Od} těch dob, co jsou tam \Ch{D}{trampové,} \Ch{E}{nesmí} z hradu \Ch{A7}{pryč,}

\Ch{D}{a tak} dole v podhradí \Ch{A7}{se šerifem} dovádí, \Ch{D}{on ji} sebral \Ch{A7}{od komnaty} \Ch{D}{klíč.}
\kr

\zs
Jednoho dne z rána roznesla se zpráva, že byl Okoř vykraden.

Nikdo neví dodnes, kdo to tenkrát odnes, nikdo nebyl dopaden.

Šerif hrál celou noc mariáš s Bílou paní v kostnici,

místo, aby hlídal, zuřivě ji líbal, dostal z toho zimnici.
\ks

\zr \kr

\kp
