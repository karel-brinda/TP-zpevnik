% -*-coding: utf-8 -*-

\zp{Rosa na kolejích}{Wabi Daněk}
\in{TP2016}{65}
\in{TP2017}{60}



\zs
<C>Tak, jako jazyk <F6>stále n<F#6>ará<G6>ží

na vylomený <C>zub,

tak se vracím k <F6>svýmu ná<F#6>dra<G6>ží,

abych šel zas <C>dál.

Přede mnou <F6>stíny se <G6>plouží

a <Ami>nad krajinou <F#dim>krouží

podivnej <F6>pták, <F#6> <G6>pták nebo <C>mrak.
\ks

\zr
Tak do toho <F6>šlápni, ať <G6>vidíš kousek <C>světa,

vzít do dlaní <F6>dálku <G6>zase jednou <C>zkus,

telegrafní <F6>dráty <G6>hrajou ti už <C>léta

to nekonečně <F6>dlou<F#6>hý <G6>mono<F#6>tón<F6>ní <C>blues.

Je ráno, je ráno.

/: Nohama <F6>stí<F#6>ráš <G6>rosu na <F#6>ko<F6>le<C>jích. :/
\kr

\zs
Pajda dobře hlídá pocestný, co se nocí toulaj,

co si radši počkaj, až se stmí, a pak šlapou dál,

po kolejích táhnou bosí a na špagátu nosí

celej svůj dům -- deku a rum.
\ks

\zr \kr

\kp
