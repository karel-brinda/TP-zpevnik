% -*-coding: utf-8 -*-

\zp{Tři citrónky}{(neznámý)}

\zs
V \Ch{C}{jedné} \Ch{Ami}{mořské} \Ch{Dmi}{pusti}\Ch{G7}{ně}
\Ch{C}{ztroskotal} \Ch{Ami}{parník} v \Ch{Dmi}{hlubi}\Ch{G7}{ně,}

\Ch{C}{jenom} tři \Ch{Ami}{malé} \Ch{Dmi}{citrón}\Ch{G7}{ky}
zůstaly na hladi\Ch{C}{ně.}
\ks

\zr
\Ch{C}{Ry}baroba \Ch{Ami}{rybaroba} \Ch{Dmi}{rybaroba} \Ch{G7}{čuču,}

\Ch{C}{rybaroba} \Ch{Ami}{rybaroba} \Ch{Dmi}{rybaroba} \Ch{G7}{čuču,}

zůstaly na hladi\Ch{C}{ně}
\kr

\zs
Jeden z nich povídá: \uv{Přátelé, netvařte se tak kysele!

Vždyť je to přece veselé, že nám patří moře celé!}
\ks

\zr  ... že nám patří moře celé. \kr

\zs
A tak se citronky plavily dál, jeden jim k tomu na kytaru hrál.

A tak se plavily do dáli až na ostrov korálový.
\ks

\zr  ... až na ostrov korálový. \kr

\zs
Tam je však stihla nehoda zlá, byla to mořská příšera.

Sežrala citrónky i s kůrou a skončila tak baladu mou.
\ks

\zr  ... a skončila tak baladu mou. \kr

\kp























