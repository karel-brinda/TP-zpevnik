% -*-coding: utf-8 -*-

\zp{Pohoda}{Samson Lenk}

\zs
U nás v \Ch{C}{ulici} bydlel pan \Ch{Emi7}{Svoboda}, 

že se \Ch{Ami}{pořád} smál, měl přezdívku \Ch{C}{Pohoda}, 

když jsem \Ch{Dmi7}{se ho ptal,} jakpak \Ch{G7}{dneska} je, 

říkával: \uv{\Ch{C}{Pohoda.}} 

Měl \Ch{C}{malej} byt, ženu a \Ch{Emi7}{dvě děti,}

v pět vstával \Ch{Ami}{do práce,} domů šel v \Ch{C}{půl} třetí, 

noviny \Ch{Dmi7}{pod paží,} cukroví \Ch{G7}{pro} děti, 

zkrátka \Ch{C}{pohoda}. 
\ks
\zr
\Ch{Dmi7}{Po domě} se \Ch{G7}{povídalo:} 

\Ch{C}{ten} má \Ch{C/H}{asi} \Ch{Ami}{příjem,}\Ch{Ami/G}{} 

\Ch{Dmi7}{zatímco} my \Ch{G7}{nadáváme}, \Ch{C}{vese}\Ch{C/H}{le} si \Ch{Ami}{žije,}\Ch{Ami/G}{}

a \Ch{Dmi7}{hlavně} paní \Ch{G7}{Šulcová} ze \Ch{C}{soused}\Ch{C/H}{ního} \Ch{Ami}{bytu}\Ch{Ami/G}{}

\Ch{D7}{} \Ch{Ami/G}{poslouchala} za dveřmi a \Ch{F}{záviděla} v \Ch{G7}{skrytu}. 
\kr
\zs
Paní Šulcová psala dopisy, 

co jsou zvláštní tím, že nemaj' podpisy, 

no a ty, úřade, no a ty počti si, 

kdo, kde a kolik.

Dík paní Šulcový, jejímu dopisu,

rychle pan Svoboda dostal se do spisu, 

ač není podpisu, je třeba přešetřit, 

autor se bojí. 
\ks
\zr
Na Svobodu začali se ptát neznámí páni, 

jak u nás ve vchodu, tak u něj v zaměstnání, 

a i když se nic nenašlo a Svoboda je čistej, 

lepší na něj dávat pozor, kdo si má být jistej. 
\kr
\zs
Včera se stěhoval od nás pan Svoboda 

na nějakou samotu někde u Náchoda 

a já vím určitě, není to náhoda, 

a tak trochu se bojím, 

že brzo i my budem hledat samoty, 

bydlení na stálo, nejen o soboty, 

kvůli pár Šulcovejm, no a vím na tuty: 

o to nestojím. 
\ks
\zr
Usměvavej Svoboda mi totiž hrozně chybí, 

zjišťuji to ponenáhlu a málo se mi líbí, 

že Šulcová je na koni, a proto není náhoda, 

že \Ch{Dmi7}{lidi zdravím} úsměvem a \Ch{G7}{odpovídám}: 

Pohoda...
\kr
\kp

