% -*-coding: utf-8 -*-

\zp{Jacek}{Jarek Nohavica}
\zs
\Ch{G}{Na} druhém břehu řeky \Ch{D}{Olzy} žije Jacek, 

\Ch{C}{mám} k němu stejně blízko \Ch{G}{jak} on ke mně, 

\Ch{G}{máváme} na sebe z \Ch{D}{říční} navigace, 

\Ch{C}{dva} spojenci a dvě \Ch{G}{spřátelené} země, 

jak malí kluci hážem z \Ch{D}{břehů} žabky, 

\Ch{C}{kdo} vyhraje, má z protěj\Ch{G}{šího} srandu, 

hlavama kroutí česko-\Ch{D}{polské} babky, 

\Ch{C}{děláme} prostě vlastní \Ch{G}{propagandu}. 

\Ch{G}{Na} na \Ch{D}{na} na \Ch{C}{na} na \Ch{G}{ná} na\Ch{D}{ na} \Ch{C}{náná}\Ch{G}{na} 
\ks

\zs
Na mostě přátelství se tvoří dlouhé fronty 

všelikých věcí za všelikou cenu, 

já mám však na to velmi úzké horizonty 

a Jacek velmi nenáročnou ženu, 

týden co týden z břehů navigace 

na sebe řveme:"Chlapče, hlavu vzhůru!", 

jak je to krásné, moci vykašlat se 

na celní předpisy a na cenzuru, 
\ks

\zs
Z Piastovské věže na nás mává kníže Měšek 

a směje se, až třepe se mu brada, 

ve zprávách večer běží horký dnešek, 

aspoň se máme s Jackem o co hádat, 

on tvrdí svoje, já zas tvrdím svoje 

a domluvit se někdy bývá marno, 

tak spolu vedem pohraniční boje 

a v praxi demonstrujem Solidarnosc, 
\ks

\zs
Na druhém břehu řeky Olše žije Jacek, 

mám k němu stejně blízko jak on ke mně, 

máváme na sebe z říční navigace, 

dva spojenci a dvě spřátelené země 

/: a voda plyne, plyne, plyne dlouhé věky, 

řeka se kroutí jako modrá šňůrka 

a my dva hážem kachnám vprostřed řeky 

krajíčky chleba o dvou stejných kůrkách. :/ 

Na na na ...
\ks
\kp





