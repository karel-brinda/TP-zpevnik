% -*-coding: utf-8 -*-

\zp{Pražce}{Pavel Dobeš}

\zs
<C>Házím tornu na svý záda, feldflašku a <G7>sumky,

navštívím dnes kamaráda z železniční průmky.
\ks

\zr (po každé sloce)

Vždyť je <C>jaro, zapni si kšandy,
pozdravuj <G7>vlaštovky a muziko, ty <C>hraj.
\kr

\zs
Vystupuji z vlaku, který mizí v dálce,

stojím v České Třebové a všude kolem pražce.
\ks

\zs
Pohostil mě slivovicí, představil mě Mařce,

posadil mě na lavici z dubového pražce.
\ks

\zs
Provedl mě domem -- nikde kousek zdiva,

všude samej pražec, jen Máňa byla živá.
\ks

\zs
Plakáty nás informují: \uv{Přijď pracovat k dráze,

pakliže ti vyhovují rychlost, šmír a saze,}
\ks

\zs
A jestliže jsi labužník a přes kapsu se praštíš,

upečeš i krávu na železničních pražcích.
\ks

\zs
A naučíš se skákat tak, jak to umí vrabec,

když na nohu si pustíš železniční pražec.
\ks

\zs
Když má děvče z Třebové rádo svého chlapce,

posílá mu na vojnu železniční pražce.
\ks

\zs
A když děti zlobí, tak hned je doma mazec,

Děda Mráz jim nepřinese ani jeden pražec.
\ks

\zs
Před děvčaty z Třebové chlubil jsem se silou,

pozvedl jsem pražec, načež odvezli mě s kýlou.
\ks

\zs
Pamatuji pouze ještě operační sál,

pak praštili mě pražcem a já jsem tvrdě spal
\ks

\zr
a bylo jaro, zapni si kšandy,
lítaly vlaštovky a zelenal se háj.
\kr

\kp
