% -*-coding: utf-8 -*-

\zp{Kaťuša}{Alexandrovci}


\zs

%<Dmi>Расцветали яблони и <A7>груши, поплыли туманы над 
%<Dmi>рекой,

%/: <Dmi>Вы<B>хо<F>дила <Gmi>на берег Ка<Dmi>тюша,
%<Gmi>на вы<Dmi>сокий <A7>берег на кру<Dmi>той. :/

<Dmi>Razcvětaly jabloni i <A7>gruši, poplyli tumany nad 
re<Dmi>koj,

/: <Dmi>vy<B>cha<F>dila <Gmi>na běreg Ka<Dmi>ťuša, 
<Gmi>na vy<Dmi>sokij <A7>běreg, na kru<Dmi>toj. :/
\ks

\zs
%Выходила, песню заводила про степного сизого орла,

%Про того, которого любила, про того, чьи письма берегла.

Vychadila, pěsňu zavadila pro stěpnavo, sizavo orla.

/: Pro tavo, katorovo ljubila, pro tavo, či pisma běregla. :/
\ks

\zs
%Ой ты, песня, песенка девичья, ты лети за ясным солнцем вслед:

%И бойцу на дальнем пограничье от Катюши передай привет.

Oj, ty pěsňa, pěseňka děvičja, ty leti za jasnym solncem vslěd.

/: I bajcu na dalněm pagraničje ot Kaťuši pěredaj privět. :/
\ks

\zs
%Пусть он вспомнит девушку простую, пусть услышит, как она поет, 

%Пусть он землю бережет родную, а любовь Катюша сбережет. 

Pusť on vzpomnit děvušku prastuju, pusť uslyšit, kak ona pajot.

/: Pusť on zemlju berežjot radnuju a ljubov Kaťuša sběrežjot. :/
\ks

\zs
= 1.
\ks

\kp



