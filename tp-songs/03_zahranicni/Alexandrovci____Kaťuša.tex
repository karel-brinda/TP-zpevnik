% -*-coding: utf-8 -*-

\zp{Kaťuša}{Alexandrovci}


\zs

%\Ch{Dmi}{Расцветали} яблони и \Ch{A7}{груши,} поплыли туманы над 
%\Ch{Dmi}{рекой,}

%/: \Ch{Dmi}{Вы-}\Ch{B}{хо-}\Ch{F}{дила} \Ch{Gmi}{на берег} Ка\Ch{Dmi}{тюша,}
%\Ch{Gmi}{на вы}\Ch{Dmi}{сокий} \Ch{A7}{берег на} кру\Ch{Dmi}той. :/

\Ch{Dmi}{Razcvětaly} jabloni i \Ch{A7}{gruši,} poplyli tumany nad 
re\Ch{Dmi}{koj,}

/: \Ch{Dmi}{vy-}\Ch{B}{cha}\Ch{F}{dila} \Ch{Gmi}{na běreg} Ka\Ch{Dmi}{ťuša,} 
\Ch{Gmi}{na vy}\Ch{Dmi}{sokij} \Ch{A7}{běreg, na} kru\Ch{Dmi}{toj.}  :/
\ks

\zs
%Выходила, песню заводила про степного сизого орла,

%Про того, которого любила, про того, чьи письма берегла.

Vychadila, pěsňu zavadila pro stěpnavo, sizavo orla.

/: Pro tavo, katorovo ljubila, pro tavo, či pisma běregla. :/
\ks

\zs
%Ой ты, песня, песенка девичья, ты лети за ясным солнцем вслед:

%И бойцу на дальнем пограничье от Катюши передай привет.

Oj, ty pěsňa, pěseňka děvičja, ty leti za jasnym solncem vslěd.

/: I bajcu na dalněm pagraničje ot Kaťuši pěredaj privět. :/
\ks

\zs
%Пусть он вспомнит девушку простую, пусть услышит, как она поет,    

%Пусть он землю бережет родную, а любовь Катюша сбережет.   

Pusť on vzpomnit děvušku prastuju, pusť uslyšit, kak ona pajot.

/: Pusť on zemlju berežjot radnuju a ljubov Kaťuša sběrežjot. :/
\ks

\zs
= 1.
\ks

\kp



