% -*-coding: utf-8 -*-

\zp{František}{Buty}

\zs
\Ch{G}{Na} hladinu rybníká svítí sluníč\Ch{C}{ko}

\Ch{Emi}{a ko}lem stojí v hustém kruhu \Ch{G}{topoly,}

\Ch{Ami}{které} tam zasadil jeden hodný \Ch{Hmi}{člověk,}

\Ch{Ami}{jmenoval} se František \Ch{D}{Dobrota.}

\ks \zs
František Dobrota, rodák z blízké vesnice,

měl hodně dětí a jednu starou babičku,

která, když umírala, tak mu řekla: \uv{Františku,

teď dobře poslouchej, co máš všechno udělat!}
\ks

\zr
3× /: \Ch{C}{Balabambam}, balabambam \Ch{C}{} \Ch{D}{} \Ch{C}{} :/

... \Ch{Ami}{a kolem} rybníka nahusto nasázet \Ch{D}{topoly.}
\kr

\zs
František udělal všechno, co mu řekla,

a po snídani poslal děti do školy,

žebřiňák s nářadím dotáhl od chalupy k rybníku,

vykopal díry a zasadil topoly.

\ks \zs
Od té doby vítr na hladinu nefouká,

takže je klidná jako velké zrcadlo,

sluníčko tam svítí vždycky rádo,

protože tam vidí Františkovu babičku.
\ks

\kp







