% -*-coding: utf-8 -*-

\zp{František}{Buty}

\zs
<G>Na hladinu rybníká svítí sluníč<C>ko

<Emi>a kolem stojí v hustém kruhu <G>topoly,

<Ami>které tam zasadil jeden hodný <Hmi>člověk,

<Ami>jmenoval se František <D>Dobrota.

\ks \zs
František Dobrota, rodák z blízké vesnice,

měl hodně dětí a jednu starou babičku,

která, když umírala, tak mu řekla: \uv{Františku,

teď dobře poslouchej, co máš všechno udělat!}
\ks

\zr
3× /: <C>Balabambam, balabambam <C> <D> <C> :/

... <Ami>a kolem rybníka nahusto nasázet <D>topoly.
\kr

\zs
František udělal všechno, co mu řekla,

a po snídani poslal děti do školy,

žebřiňák s nářadím dotáhl od chalupy k rybníku,

vykopal díry a zasadil topoly.

\ks \zs
Od té doby vítr na hladinu nefouká,

takže je klidná jako velké zrcadlo,

sluníčko tam svítí vždycky rádo,

protože tam vidí Františkovu babičku.
\ks

\kp







