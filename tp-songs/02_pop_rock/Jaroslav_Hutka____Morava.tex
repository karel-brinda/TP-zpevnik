% -*-coding: utf-8 -*-

\zp{Morava}{Jaroslav Hutka}

\zs
<Ami>Brněnská radnice stavěná do vejš<E>ky,

v ulicích práší se a nejvíc na Čes<Ami>ký <E7>

<Ami>Dětičky s jásotem z radnice máva<Dmi>jí,

<C>holubi nad měs<E>tem poletu<Ami>jí.
\ks

\zr
<G7>A Mora<C>va je krásná zem, na její <G7>slávu připijem,

só na ní hezká děvčata tvářičky <C>majó ze zlata.

Dobrý člo<Dmi>věk ještě ži<E>je: na Moravě víno pi<Ami>je.

Proto ať vzkvétá Mora<C>va a je<G7>jí slá<C>va. <E>
\kr

\zs
V Špilberku na kopci v domečku kamenným divěj se cizinci jakejs' byl na ženy.

Děti se poučí dobou fe-udální -- silný slabé mučí, to dnes už není.
\ks

\zr\kr

\zs
A brněnský kolo, docela dřevěný, letos prohánělo motorky závodní.

Dětičky mávaly Frantovi Šťastnýmu, na dráze spatřily samou šmouhu.
\ks

\zr\kr

\zs
Copak to vyrostlo na Královým poli, už to snad uzrálo, že to nesklidili?

Dětičky mávají komínům továrním, radostně vdechují ten jejich dým.
\ks

\zr\kr

\zs
No a brněnskej drak v Nilu se narodil, dětem nahání strach, má jméno Krokodil.

Visí na řetizku, v průvanu houpe se, píšu do notysku, ať nekrmí se.
\ks

\zr\kr

Rec: <Ami> \uv{Tož, dámo, co si dáte?}

<E> \uv{Nic nechcu, pane vrchní, su smutná.}

<E> \uv{Tož něco si dát musíte, jste v jedničce.}

<E> <Ami> \uv{Říkám, nic nechcu, su smutná!}

<Ami> \uv{Tož si dajte něco tvrdýho, <Dmi> třeba gruziňak.}

\uv{<Ami>Nechcu, pane vrchní, <E7>po tem su <Ami>smutná.}

\zr\kr

\kp
