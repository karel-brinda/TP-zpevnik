% -*-coding: utf-8 -*-

\zp{Strom kýve pahýly}{film Pražská 5}

\zs
\Ch{A}{Když} slunce \Ch{Hmi7}{zapadá,} tak \Ch{A}{moje} nála\Ch{D}{da} \Ch{E7}{kles}\Ch{A}{á,} \Ch{Hmi7}{} \Ch{A}{} \Ch{D}{} \Ch{E7}{}

\Ch{A}{strom} kýve \Ch{Hmi7}{větvemi,} pří\Ch{A}{telem} on je \Ch{D}{mi,} \Ch{E7}{ple}\Ch{A}{sá,} \Ch{Hmi7}{} \Ch{A}{} \Ch{D}{} \Ch{E7}{}

\Ch{A}{já} však mám v duši žal, čert \Ch{Hmi7}{ví, kde} se tam vzal,

te\Ch{A}{pe,} te\Ch{Hmi7}{pe, te}\Ch{A}{pe,} te\Ch{D}{pe.} \Ch{E7}{}
\ks

\zr
\Ch{A}{Strom} kýve \Ch{Hmi7}{pahýly,} chtěl \Ch{A}{bych} jen na chví\Ch{D}{li te}be,

\Ch{A}{strom} kýve \Ch{Hmi7}{pahýly,} chtěl \Ch{A}{bych} jen na chví\Ch{D}{li te}be,

\Ch{A}{rosu} mám v \Ch{Hmi7}{kanadách,} v mých \Ch{A}{černých} kana\Ch{D}{dách} zebe,

\Ch{A}{rosu} mám v \Ch{Hmi7}{kanadách,} v mých \Ch{A}{černých} kana\Ch{D}{dách} zebe.
\kr

\zs
Znám dobře kůru lip, té dal jsem kdysi slib mlčení,

znám řeč, jíž mluví hřib, sám jako jedna z ryb jsem němý,

však lesy, ty mám rád, tam cítím se vždycky mlád,

vždycky líp, vždycky líp, vždycky líp, vždycky líp.
\ks

\zr
\kr

\kp



