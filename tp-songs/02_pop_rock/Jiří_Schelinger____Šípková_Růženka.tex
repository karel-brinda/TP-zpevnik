% -*-coding: utf-8 -*-

\zp{Šípková Růženka}{Jiří Schelinger}

\zs
Ještě \Ch{Gmi}{spí a spí} a spí zámek \Ch{F}{šípkový,}

\Ch{Gmi}{žádný} princ tam v lesích \Ch{Dmi}{ptáky} neloví.

Ještě \Ch{Gmi}{spí a sp}í a spí dívka \Ch{F}{zakletá,}

\Ch{Gmi}{u lůž}ka jí planá \Ch{Dmi}{růže} rozkvétá.  \Ch{D}{ } \Ch{C}{ } \Ch{B}{ }

To se \Ch{C}{schválně} dětem \Ch{Gmi}{říká,}  \Ch{G}{ } \Ch{F}{ } \Ch{D#}{ }

aby s \Ch{F}{důvěrou} šly \Ch{B}{spát}, klidně \Ch{Dmi}{spát,}

že se \Ch{Gmi}{dům} probou\Ch{F}{zí}

a ta \Ch{B}{kráska} proci\Ch{D#}{tá,} \Ch{D#}{ } \Ch{D}{ }

\Ch{Cmi}{zatím} spí tam \Ch{Dmi}{dál,} spí tam v \Ch{Gmi}{růžích.}
\ks

\zs
Kdo jí ústa k ústům dá, kdo ji zachrání,

kdo si dívku pobledlou vezme za paní?

Vyjdi zítra za ní a nevěř pohádkám,

žádny princ už není, musíš tam jít sám.

To se schválně dětem říká,

aby s důvěrou šly spát, klidně spát,

že se dům probouzí a ta kráska procitá,

zatím spí tam dál, spí v růžích.
\ks

\zs
\Ch{Gmi}{Musíš} přijít \Ch{F}{sám},

nesmíš \Ch{B}{věřit} pohád\Ch{D#}{kám,} \Ch{D#}{ } \Ch{D}{ }

/: \Ch{Cmi}{čeká} dívka \Ch{F}{dál}, spí tam v \Ch{Dmi}{růžích.} :/
\ks

\kp






















