% -*-coding: utf-8 -*-

\zp{Bitva o Karlův most}{Michal Prokop}

\zs
\Ch{C}{Má} vlasy \Ch{G}{dlouhý} do půl \Ch{C}{pasu,}

{k tur}istům \Ch{F}{jistou} \Ch{C}{náklonnost,}

{barokní} \Ch{G}{sochy} v letním \Ch{C}{jasu}

{sedí} tu \Ch{F}{jako} hejno \Ch{C}{vos.}

/: \Ch{Ami}{Má vl}asy \Ch{Emi}{do pas}u a \Ch{F}{jen} tak pro \Ch{C}{radost}

{nabízí} \Ch{G}{ortel} nebo \Ch{C(F)}{spásu} :/ v bitvě o Karlův \Ch{C}{most.}
\ks

\zr
\Ch{Emi}{To pro} te\Ch{F}{be} král Karel \Ch{C}{Čtvrtý,}

\Ch{F}{má} \Ch{Emi}{lásko} vlasa\Ch{Dmi}{tá,}

/: \Ch{G}{dáv}\Ch{C}{al} do malty \Ch{F}{žloutek}

a mrhal \Ch{Dmi}{úsilím} a \Ch{C}{pruty} \Ch{F}{ze} zla\Ch{C}{ta.} :/
\kr

\zs
Má vlasy dlouhý do půl pasu,

svůj bledej šampónovej chvost,

a svádí právě toho času

bitvu o Karlův most.


/: Má v očích muškety, děla a čeká na Švéda,

klíč ke zbrojnici těla :/ až na Kampě mu dá.
\ks


\zr \kr

\zs

Máš vlasy dlouhý do půl pasu,

tvou svatou válku vidí most,

barokní sochy v letním jasu

a Němců kolem jako vos.



Máš vlasy do pasu a jen tak pro radost

nabízíš ortel nebo spásu

3× /: v bitvě o Karlův most. :/
\ks

\kp





