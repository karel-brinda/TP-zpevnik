% -*-coding: utf-8 -*-

\zp{Bitva o Karlův most}{Michal Prokop}

\zs
<C>Má vlasy <G>dlouhý do půl <C>pasu,

{k tur}istům <F>jistou <C>náklonnost,

{barokní} <G>sochy v letním <C>jasu

{sedí} tu <F>jako hejno <C>vos.

/: <Ami>Má vlasy <Emi>do pasu a <F>jen tak pro <C>radost

{nabízí} <G>ortel nebo <C(F)>spásu :/ v bitvě o Karlův <C>most.
\ks

\zr
<Emi>To pro te<F>be král Karel <C>Čtvrtý,

<F>má <Emi>lásko vlasa<Dmi>tá,

/: <G>dáv<C>al do malty <F>žloutek

a mrhal <Dmi>úsilím a <C>pruty <F>ze zla<C>ta. :/
\kr

\zs
Má vlasy dlouhý do půl pasu,

svůj bledej šampónovej chvost,

a svádí právě toho času

bitvu o Karlův most.


/: Má v očích muškety, děla a čeká na Švéda,

klíč ke zbrojnici těla :/ až na Kampě mu dá.
\ks


\zr \kr

\zs

Máš vlasy dlouhý do půl pasu,

tvou svatou válku vidí most,

barokní sochy v letním jasu

a Němců kolem jako vos.



Máš vlasy do pasu a jen tak pro radost

nabízíš ortel nebo spásu

3× /: v bitvě o Karlův most. :/
\ks

\kp





