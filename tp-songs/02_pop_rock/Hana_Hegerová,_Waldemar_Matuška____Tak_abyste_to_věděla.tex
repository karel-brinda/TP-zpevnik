% -*-coding: utf-8 -*-

\zp{Tak abyste to věděla}{Hana Hegerová, Waldemar Matuška}
\Ch{Gmi}{~~} \Ch{Gmi}{~~}
\Ch{D7}{~~} \Ch{D7}{~~}
\Ch{Gmi}{~~} \Ch{Gmi}{~~}
\Ch{D7}{~~} \Ch{D7}{~~}

\zs
\Ch{Gmi}{Marně} si hlavu \Ch{Cmi}{lámu,}
\Ch{D7}{proč} muži větši\Ch{Gmi}{nou}

neradi vidí \Ch{Cmi}{dámu}
\Ch{A7}{chladnou} a nečin\Ch{D7}{nou.}

\Ch{Cmi}{Nedávno} přišel \Ch{D7}{ke mně,}
\Ch{Cmi}{sladkej} a \Ch{A7}{milej} \Ch{D7}{byl}

\Ch{Gmi}{a doop}ravdy \Ch{A7}{jemně},
\Ch{D7}{jemně} mě oslo\Ch{G}{vil.}
\ks

\zr
\Ch{G}{Vy} byste porád seděla
a nevydala \Ch{D7}{hlásku}

a to se přece nedělá
a já vás varu\Ch{G}{ju!}

Tak \Ch{G7}{abys}te to \Ch{C}{věděla,}
tak \Ch{Eb7}{já vám} vyznám \Ch{G}{lásku,}

Tak a-\Ch{F#7}{by-}\Ch{F7}{ste} \Ch{E7}{to} \Ch{A7}{věděla,}
tak \Ch{D7}{já v}ás \Ch{Gmi}{miluju!}

\Ch{D7}{~~} \Ch{D7}{~~} \Ch{Gmi}{~~} \Ch{Gmi}{~~} \Ch{D7}{~~} \Ch{D7}{~~}
\kr

\zs
Já nemám ráda muže,
kteří se vnucujou,

za prachy oni u žen
si štěstí kupujou.

To nejsou muži pro mne,
mě nejspíš získá si

ten, který přijde skromně
a tiše prohlásí:
\ks

\zr \kr

\zs
Ten, který přišel včera
do naší ulice

za večerního šera,
byl hezkej velice,

oči mu něžně planou,
jak se tak plaše ptá,

a když mu řeknu ano,
tak tiše zašeptá:
\ks

\zr

...\Ch{D7}{jááá} vás \Ch{D7}{milu}ju\Ch{G}{}

á dá \Ch{D7}{dá} já jaš ta \Ch{G}{dá}
\kr
\kp


