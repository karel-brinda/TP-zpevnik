% -*-coding: utf-8 -*-

\zp{Dívka s perlami ve vlasech}{Aleš Brichta}
\in{TP2016}{73}
\in{TP2017}{66}

\zs
<Emi>Zas mě tu <D>máš, <Ami>nějak se <Emi>mračíš,
vybledlej <D>smích <Ami>už ve dve<Emi>řích.

S čelenkou z <D>perel <Ami>svatozář <Emi>ztrácíš,
kolik se <D>platí <Ami>za vláčnej <Emi>hřích?
\ks

\zr
No tak, <G>lásko, <D>co chceš mi říct?
<Ami>Máš už perly, <Emi>možná i víc.

<G>Lásko, <D>na co se ptáš? <Ami>Svíčku zhasí<Emi>náš.

Nemám, <G>lásko, <D>co bych ti dal,
<Ami>chtěla's všechno, <Emi>nebyl jsem král.

<G>Lásko, <D>na co se ptáš? <Ami>Perly ve vlasech <Emi>máš.
\kr

\zs
Tvý horký rty, víc radši ne,
nejsou už mý, nejsi ma chère.

Něco snad chápu, to ne, to ne,
bolí to, hořím jak černej tér.
\ks

\zr
No tak, lásko, kdo mi tě vzal,
komu's dala tělo i duši?

Lásko, na co je pláč,
když to skončit má?

Vždyť už, lásko, svý perly máš,
tak proč padaj', měněj' se v slzy?

Lásko, na co je pláč?
Perly ve vlasech máš.
\kr

\zr No tak, lásko, co chceš mi říct?... \kr

\zs
Chtěla jsi víc pro svoje touhy,
já, chudej princ, mám jen, co mám.

Co vlastně zbývá? Jen slzy pouhý
ze svatebních zvonů, z nebeskejch bran.
\ks

\zr No tak, lásko, kdo mi tě vzal?... \kr

\zr No tak, lásko, co chceš mi říct?... \kr

\zr No tak, lásko, kdo mi tě vzal?... \kr

\kp
