% -*-coding: utf-8 -*-

\zp{Štěstí je krásná věc}{Richard Müller}

\zs
<C>Například východ slunce <Emi7>a vítr ve vět<F>vích <Dmi> <G7>

<C>anebo píseň tichou <Emi7>jak padající <F>sníh, <Dmi> <G>

<Dmi>tak to prý nelze <G7>koupit za <C>žádný pení<F>ze,

jenže <C>zbejvá spousta <F>věcí, <C>a ty <D7>koupit <G>lze.
\ks

\zs
<G7>Jó, vždyť <C>víš, <Emi>štěstí je krásná <F>věc, <Dmi> <G7>

vždyť <C>víš, <Emi>štěstí je krásná <F>věc. <Dmi> <G>

<Dmi>Štěstí je tak <G>krásná a <C>přepychová <F>věc,

ale <C>prachy si <G7>za něj nekou<C>píš. <G7>
\ks

\zs
Jó, kartón Marlborek a taky porcelán

s modrejma cibulkama, tapety, parmazán,

pět kilo uheráku nebo džínsy Calvin Klein,

tak možná že to není štěstí, ale je to fajn.
\ks

\zr
Například východ slunce a vítr ve větvích

anebo píseň tichou jak padající sníh,

tak to prý nelze koupit za žádný peníze,

jenže zbejvá spousta věcí, a ty koupit lze.
\kr

\zs
Takovej východ slunce je celkem v pořádku,

peníze mám ale radši, ty stojej za hádku.

A proto když se dočtu o zemětřesení

nebo o bouračce, no tak řeknu: \uv{K neuvěření.}
\ks

\zr2× \kr

\kp
