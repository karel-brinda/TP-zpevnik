% -*-coding: utf-8 -*-

\zp{Kolej Yesterday}{Michal Prokop}

\zs
To \Ch{Dmi}{bejvávaly} \Ch{C}{na koleji} \Ch{Dmi}{časy,}
už ráno \Ch{C}{začal} večí\Ch{F}{rek,}

\Ch{Eb}{někdo} dal dvacet, někdo \Ch{Dmi}{stovku}
\Ch{Eb}{a sta}věli jsme Eiffe\Ch{Dmi}{lovku}

ze snů a \Ch{B}{peří,} \Ch{C}{ze} si\Ch{Dmi}{rek.}
\ks

\zs
To tenkrát ještě holkám rostly vlasy,
to bylo před tím vejbuchem.

/: A kdo měl žízeň, tak ji hasil, :/
vracel se s kytkou za uchem.
\ks

\zr
\Ch{A}{Prázdnou} \Ch{D}{ulicí} \Ch{C}{na ko}\Ch{G}{lej,}

jo, to je \Ch{C}{dávno}, yester\Ch{F}{day,} Yesterday.
\Ch{Emi}{} \Ch{A}{} \Ch{Dmi}{}
\kr

\zs
To bejvávalo na koleji snění,
bramborák voněl v kuchyni,

někdo dal dvacet, někdo pade,
za to jsme měli, kamaráde,

hned štěně piva ve skříni.
\ks

\zs
Tenkrát, hochu, ještě ozáření
hrozilo leda od slunce

/: a byly řeky po setmění, :/ kam jsi moh' chodit na sumce.
\ks

\zr
A ty pak hodit na olej,

jo, to je dávno, yesterday, Yesterday.
\kr

\zs
To bejvávaly na koleji časy,
už ráno začal večírek,

někdo dal dvacet, někdo stovku
a někdo zbořil Eiffelovku

ze snů a peří, ze sirek.
\ks

\zr
A od tý doby ta kolej

patří už dávno yesterday...

Yesterday, I believe in yesterday.
\kr

\kp
