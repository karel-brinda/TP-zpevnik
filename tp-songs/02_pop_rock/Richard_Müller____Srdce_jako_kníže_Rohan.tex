% -*-coding: utf-8 -*-

\zp{Srdce jako kníže Rohan}{Richard Müller}

\zs
\Ch{F}{Měsíc} je jak Zlatá bula \Ch{C}{Sicilská,}
\Ch{Ami}{stvrzuje} se, že kdo chce, se \Ch{G}{dopíská,}

\Ch{F}{pod} lampou jen krátce, v přítmí \Ch{C}{dlouze} zas,
\Ch{Ami}{otevře} ti Kobera a \Ch{G}{můžeš} mezi nás.
\Ch{F} {} \Ch{C}{} \Ch{Ami}{} \Ch{G}{} \Ch{F}{} \Ch{C}{} \Ch{Ami}{} \Ch{G}{}
\ks

\zs
\Ch{F}{Moje} teta, tvoje teta, \Ch{C}{parole,}
\Ch{Ami}{dvaatřicet} karet křepčí \Ch{G}{na} stole,

\Ch{F}{měsíc} svítí sám a chleba \Ch{C}{nežere,}
\Ch{Ami}{ty to} ale koukej trefit, \Ch{G}{frajere,} protože...
\ks

\zr
\Ch{F}{Dnes} je valcha u starýho \Ch{C}{Růžičky,}
\Ch{Ami}{dej si} prachy do pořádný \Ch{G}{ruličky,}

\Ch{F}{co} je na tom, že to není \Ch{C}{extra} nóbl byt,
\Ch{Ami}{srdce} jako kníže Rohan \Ch{G}{musíš} mít.

\Ch{F}{} \Ch{C}{} \Ch{Ami}{} \Ch{G}{}
\kr

\zs
Ať si přes den docent nebo tunelář,
herold svatý pravdy nebo jinej lhář,

tady na to každej kašle zvysoka,
pravda je jen jedna -- slova proroka říkaj, že:
\ks

\zr
Když je valcha u starýho Růžičky,
budou v celku nanic všechny řečičky,

buďto trefa nebo kufr -- smůla nebo šnyt,
jen to srdce jako Rohan musíš mít.
\kr

\zs
Kdo se bojí, má jen hnědý kaliko,
možná občas nebudeš mít na mlíko,

jistě ale poznáš, co jsi vlastně zač,
svět nepatřil nikomu, kdo nebyl hráč.
A proto...
\ks

\zr
Ať je valcha u starýho Růžičky,
nebo pouť až k tváři Boží rodičky,

ať je válka, červen, mlha, bouřka nebo klid,
srdce jako kníže Rohan musíš mít.
\kr

\zr
Dnes je valcha u starýho Růžičky,
když jsi malej, tak si stoupni na špičky,

malej nebo nachlapenej Cikán, Brňák, Žid,
srdce jako kníže Rohan musíš mít.
\kr

\zr
Dnes je valcha u starýho Růžičky,
dej si prachy do pořádný ruličky,

co je na tom, že to není extra nóbl byt,
srdce jako kníže Rohan musíš mít.
\kr

\zr
Ať je valcha u starýho Růžičky,
nebo pouť až k tváři Boží rodičky,

ať je válka, červen, mlha, bouřka nebo klid,
srdce jako kníže Rohan musíš mít.
\kr

\kp





