% -*-coding: utf-8 -*-

\zp{Holky z naší školky}{Standa Hložek, Petr Kotvald}

\zs
\Ch{D}{Majdalénka}, \Ch{G}{Apolénka} s \Ch{A}{Veronikou}

a taky \Ch{D}{Věrka}, Zdenka, \Ch{G}{Majka}, Lenka s \Ch{A}{Monikou,}

no jasně, \Ch{D}{Klára}, Hančí, \Ch{G}{Bára}, Mančí \Ch{A}{již} nevím čí,

to všechno \Ch{D}{byly} holky z \Ch{G}{naší} školky \Ch{A}{senzační.}
\ks

\zr
\Ch{D}{Jé,} \Ch{A}{jé,} \Ch{G}{jé,}
\Ch{A}{kdepak ty} \Ch{D}{fajn} holky \Ch{G}{jsou}
a kde \Ch{D}{maj'} hračky \Ch{G}{svý,}
ty naše \Ch{Emi7}{lásky} tříle\Ch{A}{tý?}

\Ch{D}{Pá,} \Ch{A}{pá,} \Ch{G}{pá,}
\Ch{A}{řekli jsme} \Ch{D}{pá před} škol\Ch{G}{kou},
bylo \Ch{D}{nám} právě \Ch{G}{šest}
a začla \Ch{Emi7}{další} dívčí \Ch{A}{šou.}
\kr

\zs
Ve škole Daniela, Michaela s Romanetou
a taky Adriana, Mariana se Žanetou

a hlavně príma Radka -- kamarádka, co všechno ví,

tyhlety holky byly naše víly školních dní.
\ks

\zr
Jé, jé, jé,
kdepak ty fajn holky jsou
a kde maj žákovský,
ty naše lásky klukovský?

Čau, čau, čau,
řekli jsme čau před školou,

táhlo nám na patnáct
a začla další dívčí šou.
\kr

\zs
Na gymplu bezva Šárka, třída Klárka, Táňa jak sen

a taky senza Jenka v podkolenkách, veselá jen

a všechny v sexy tričku, postavičku měly ham-ham,

no prostě príma štace, inspirace k maturitám.
\ks

\zr
Jé, jé, jé,
kdepak ty fajn holky jsou
a kde maj zazděný
naše lásky vysněný?

Au, au, au,
vzlykli jsme au, čau a pá,
už se dál nekoná
žádná dívčí školní šou.
\kr

\zs
I když pak poznali jsme spoustu dalších dívek a jmen,

plavovlásky, černovlásky, žár i sen,

v rytmu diska, z dálky, z blízka i v náručí,

přesto jsou stále holky z naší školky nejlepší.
\ks

\zr
Jé, jé, jé,
kdepak ty fajn holky jsou
a kde maj cůpky svý,
ty naše lásky tříletý?

/: Pá, pá, pá,
říkame dál před školkou,
to se ví, léta jdou,
ale ty holky nestárnou. :/
\kr

\kp
