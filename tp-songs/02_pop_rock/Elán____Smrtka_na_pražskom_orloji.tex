% -*-coding: utf-8 -*-

\zp{Smrtka na pražskom Orloji}{Elán}
\zs
\Ch{Ami}{Turisti} \Ch{Dmi}{vrabce,} ško\Ch{Ami}{ly,}

orloj a \Ch{Dmi}{apošto}\Ch{Ami}{li,}

na konci \Ch{Dmi}{smrťka} sto\Ch{Ami}{jí}

\Ch{Amimaj7}{a každy sa jej} bojí. \Ch{Ami}{}
\ks

\zs
Vyusmieva sa na mňa

tá smutná stará panna,

verí že bude moja,

jej bozky strašne bolia.
\ks

\zr
4× /: \Ch{Ami}{Óóóóóóóo}...    \Ch{Dmi}{} \Ch{Ami}{} :/

\Ch{Ami}{Tej kr}áske na \Ch{Emi}{pražskom} Orlo\Ch{Ami}{ji}

\Ch{C}{odkážte,} že o ňu \Ch{Emi}{vôbec} nesto\Ch{C}{jím,}

nech si ďalej \Ch{Emi}{máta} na ve\Ch{C}{ži,}

\Ch{Ami}{ja verím,} že \Ch{Emi}{budem} stále \Ch{Ami}{žiť.}
\kr

\zs
Smrť niečo pripomína,

čo to raz všetci zistia,

je pre každého iná

a predsa vždy tá istá.
\ks

\zr \kr

\zr
Nech si na mňa čaká každý deň,

ja jej na to rande nikdy neprídem.

Pozývam vás všetkých na pivo,

kašlem na smrť -- verím na život.
\kr

\kp



