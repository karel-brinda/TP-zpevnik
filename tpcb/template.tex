\ifdefined\SINGLE
  \def\ONESIDE{}
  \def\NOHEADER{}
\fi

\documentclass[11pt,a4paper%
  \ifdefined\ONESIDE,oneside\fi%
]{book}

\usepackage{fontspec}
\usepackage{index}
\usepackage{calc}
\usepackage{fancyhdr}
\usepackage[czech]{babel}
\usepackage[chordbk]{songbook}
\usepackage[xetex,pdfpagelabels=false,pdfborder={0 0 0}]{hyperref}
\usepackage{forloop}
\usepackage{verbatim}

\unless\ifdefined\SINGLE
  \newindex[cisloPisne]{default}{idx_pisne}{ind_pisne}{Rejstřík písní}
  \newindex[cisloPisne]{interpreti}{idx_interpreti}{ind_interpreti}{Rejstřík interpretů}
\fi

\selectlanguage{czech}

\newcounter{cisloPisne}
\newcommand\cisloPisne{\arabic{cisloPisne}}

% Hlavičky
\ifdefined\NOHEADER
  \pagestyle{empty}
\else
  \pagestyle{fancy}
  \fancyhead[RE]{\ifdefined\RHEAD\RHEAD\fi}
  \fancyhead[LO]{\ifdefined\LHEAD\LHEAD\fi}
  \fancyhead[RO]{}
  \fancyhead[LE]{}
  \fancyfoot{}
\fi

% Zkratky pro makra songbook.sty
\newcommand\zp[2]{
  \stepcounter{cisloPisne}
  \unless\ifdefined\SINGLE
    \label{pis.\cisloPisne}
  \else
    \ifnum\value{cisloPisne} > 1
      \errmessage{Více než jedna píseň v režimu SINGLE.}
    \fi
  \fi
  \begin{song}{#1}{}{}{}{#2}{}
  \twopagecheck
  \global\lastsong={#1}
  \unless\ifdefined\SINGLE
    \index{#1}
    \index[interpreti]{#2!#1}
  \fi
}
\newcommand\kp{\end{song}}
\newcommand\zs{\begin{SBVerse}}
\newcommand\ks{\end{SBVerse}}
\newcommand\zr{\begin{SBChorus}}
\newcommand\kr{\end{SBChorus}}

% Nastavení songbook
\renewcommand\CpyRt{}
\renewcommand\SBChorusTag{R:}
\renewcommand\SBBaseLang{Czech}
\renewcommand\SBUnknownTag{}
\renewcommand\SBWAndMTag{}
\font\myTinySF=cmss8 at 8pt
\renewcommand{\CpyRt}{}
\renewcommand{\SBRef}{}
\renewcommand{\SBIntro}{}
\renewcommand{\SBExtraKeys}{}

% Makra pro vkládání celých PDF (přední a zadní obálka)
\newcounter{insertCur}
\newcounter{insertTotal}
\newcommand\insertPage[2]{\shipout\vbox to 29.7cm{\vss\hbox to 21cm{\hss{\XeTeXpdffile #1 page #2 }\hss}\vss}\stepcounter{page}}
\newcommand\countPages[1]{\setcounter{insertTotal}{\XeTeXpdfpagecount #1 }}
\newcommand\insertPDF[1]{\countPages{#1}\stepcounter{insertTotal}
  \forloop{insertCur}{1}{\value{insertCur} < \value{insertTotal}}{%
    \insertPage{#1}{\value{insertCur}}}}

% Makra pro kontrolu rozložení dvoustran
\newcounter{lastpage}
\newcounter{numpages}
\newtoks\lastsong
\newtoks\errors
\newcounter{errorCount}
\def\space{ }
\newcommand\twopagecheck{%
  \unless\ifdefined\SKIPCHECK
  \unless\ifdefined\ONESIDE
    \setcounter{numpages}{\value{page}}
    \addtocounter{numpages}{-\value{lastpage}}
    \ifnum\value{numpages}>2
      \message{^^J^^J\space\space Píseň "\the\lastsong" má víc dvě stránky.^^J^^J}
      \stepcounter{errorCount}
      \edef\nerrors{\the\errors^^J\space\space\the\lastsong}
      \global\errors\expandafter{\nerrors}
    \else\ifnum\value{numpages}>1
      \unless\ifodd\value{page}
        \message{^^J^^J\space\space Píseň "\the\lastsong" začala na pravé a skončila na levé.^^J^^J}
        \stepcounter{errorCount}
        \edef\nerrors{\the\errors^^J\space\space\the\lastsong}
        \global\errors\expandafter{\nerrors}
      \fi\fi
    \fi\fi
    \setcounter{lastpage}{\value{page}}
  \fi
}

\newcommand\emptyPage{\shipout\vbox to \vsize{\hbox to \hsize{\hss}\vss}\stepcounter{page}}

\newenvironment{lilypond}{\comment}{\endcomment} % Ignorovat lilypond, pokud zbude v .tex

%%%%%%%%%%%%%%%%%%%%%%%%%%%%%%%%%%%%%%%%%%%%%%%%%%%

\begin{document}

% FIXME: co tohle dělá?
\ifWordBk
  \twocolumn
\fi

% Stránky je potřeba počítat od sudé (pravé)
\setcounter{page}{0}

% Přední obálka
\ifdefined\FRONTCOVER
  \insertPDF{\FRONTCOVER}
  \unless\ifdefined\ONESIDE
    % Pokud přední obálka existuje, vloží ji a jednu nebo dvě volné strany za 
    % ni (mimo ONESIDE), aby první stránka zpěvníku vyšla napravo
    \emptyPage
    \ifodd\value{page}\emptyPage\fi
    % Pro budoucí \twopagecheck v \zs první písničky
    \setcounter{lastpage}{\value{page}}
  \fi
\fi

%%%%%%%%%%%%%%%%%%%%%%%%%%%%%%%%%%%%%%%%%%%%%%%%%%%
% Seznam písní

\input{\thelist}

%%%%%%%%%%%%%%%%%%%%%%%%%%%%%%%%%%%%%%%%%%%%%%%%%%%

% Kontrola dvojstránek probíhá v rámci \zs, takže pro poslední písničku ji 
% musíme zavolat ručně
\twopagecheck

% Hot Fix: přebít \thispagestyle{plain} z index.sty, způsobuje čísla stránek
% na první stránce rejstříku
\renewcommand\thispagestyle[1]{}

\unless\ifdefined\SINGLE
  % Rejstřík podle interpretů
  \fancyfoot{}
  \printindex[interpreti]
  % Rejstřík podle písní
  \fancyfoot{}
  \printindex
\fi

% Zadní obálka
\ifdefined\BACKCOVER
  \unless\ifdefined\ONESIDE
    % Vložit jednu nebo dvě prázdné stránky tak, aby poslední strana zadní 
    % obálky (pokud existuje) vyšla nalevo (mimo ONESIDE)
    \emptyPage
    \countPages{\BACKCOVER}
    \addtocounter{insertTotal}{\value{page}}
    \ifodd\value{insertTotal}\emptyPage\fi
  \fi
  \insertPDF{\BACKCOVER}
\fi

% Vypsat písně, které neprošly kontrolou dvoustran
\ifnum\value{errorCount} > 0
  \errmessage{^^J^^J^^J**Chyby zarovnání dvoustran u písní:**\the\errors^^J^^J%
  Překlad nyní skončí chybou. Opravte konflikty a spusťte znovu, případně,^^J%
  jestliže chcete tuto kontrolu vypnout, přidejte options nebo options = [ "SKIPCHECK"\space]^^J%
}
\fi

\end{document}
